\documentclass{article}

\usepackage[T1]{fontenc}
\usepackage[sfdefault,light]{FiraSans}
\usepackage{fontawesome}
\usepackage{xcolor}
\usepackage{calc}

\usepackage[cm,headings]{fullpage}

\usepackage{enumitem}
\usepackage{tabularx}
\usepackage{fancyhdr}
\pagestyle{fancy}

\usepackage{comment} % Para comentários dentro de um bloco

\usepackage{hyperref}
\hypersetup{
    colorlinks=true,
    urlcolor=primary,
}
%-------------------------------------------------------------------------------------------------
%---------- CUSTOM COMMANDS
%-------------------------------------------------------------------------------------------------

%---------- Icons
\newcommand{\icon}[1]{\begin{tabular}{p{\iconWidth}}{#1}\end{tabular}}
\renewcommand{\section}[2]{\vspace{0.5em}\color{primary}\textbf{{#2}{#1}}\hrule\color{black}}

%---------- Head
\renewcommand{\headrulewidth}{0pt}
\newcommand{\cvText}[1]{#1\vspace{0.75em}}

%---------- Entries
% Entry with Name, Time, Position, Location and Details
\newcommand{\cvEntryNTPLD}[5]{
    \begin{itemize}[leftmargin=\entryMargin]
        \item[]
            \begin{tabularx}{\textwidth-\entryMargin}{Xr}
                \textbf{\color{black}#1} & {\color{secondary}\small#2} \\
                \textit{\color{secondary}\small#3} & \textit{\color{secondary}\small#4} \\
            \end{tabularx}\vspace{-0.75em}
        \begin{itemize}[labelsep=\entryMargin-1.5em,leftmargin=\entryMargin]
            \small#5
        \end{itemize}
    \end{itemize}
}
% Entry with Name, Time, Position and Location
\newcommand{\cvEntryNTPL}[4]{
    \begin{itemize}[leftmargin=\entryMargin]
        \item[]
            \begin{tabularx}{\textwidth-\entryMargin}{Xr}
                \textbf{\color{black}#1} & {\color{secondary}\small#2} \\
                \textit{\color{secondary}\small#3} & \textit{\color{secondary}\small#4} \\
            \end{tabularx}\vspace{-0.75em}
    \end{itemize}
}
% Entry with Name, Time and Details
\newcommand{\cvEntryNTD}[3]{
    \begin{itemize}[leftmargin=\entryMargin]
        \item[]
            \begin{tabularx}{\textwidth-\entryMargin}{Xr}
                \textbf{\color{black}#1} & {\color{secondary}\small#2} \\
            \end{tabularx}\vspace{-0.75em}
        \begin{itemize}[labelsep=\entryMargin-1.5em,leftmargin=\entryMargin]
            \small#3
        \end{itemize}
    \end{itemize}
}
% Entry with Name and Details
\newcommand{\cvEntryND}[1]{
    \small
    \begin{itemize}[leftmargin=\entryMargin]
        #1
    \end{itemize}
    \normalsize
}
% Entry for Languages (up to 4)
\newcommand{\cvItemL}[8]{
    \item[]
        \begin{tabularx}{\textwidth-\entryMargin}{XXXX}
            {\textbf{#1} \textit{#2}}&{\textbf{#3} \textit{#4}}&{\textbf{#5} \textit{#6}}&{\textbf{#7} \textit{#8}}
        \end{tabularx}\vspace{-0.25em}
}
% Entry for Academical Projects
\newcommand{\cvEntryAcademicalNTD}[3]{\cvEntryNTD{\educationIcon{Academical: }#1}{#2}{#3}}
% Entry for Personal Projects
\newcommand{\cvEntryPersonalNTD}[3]{\cvEntryNTD{\personalIcon{Personal: }#1}{#2}{#3}}
% Entry for Work Projects
\newcommand{\cvEntryWorkNTD}[3]{\cvEntryNTD{\workIcon{Work: }#1}{#2}{#3}}

% Bullet point under an entry's details list
\newcommand{\cvItem}[1]{\item[\bulletIcon]{#1\vspace{-0.25em}}}
% Special bullet point under an entry's details list
\newcommand{\cvItemS}[1]{\item[\accentIcon]{#1\vspace{-0.25em}}}
% Bulletless point under an entry's details list, with Name and Description 
\newcommand{\cvItemND}[2]{\item[]{\textbf{#1}\hspace{1em}#2}\vspace{-0.25em}}

%-------------------------------------------------------------------------------------------------
%---------- SETTINGS HERE
%-------------------------------------------------------------------------------------------------
%---------- Colours
\definecolor{primary}{HTML}{007BFF}
\definecolor{secondary}{HTML}{3C454D}
\definecolor{info}{HTML}{17A2B8}

%---------- Margins
\raggedbottom
\raggedright
\setlength{\tabcolsep}{0in}
\setlength{\voffset}{-0.5cm}
\setlength{\headheight}{3.5em}
\addtolength{\headsep}{-2em}
\addtolength{\oddsidemargin}{-0.25cm}
\addtolength{\evensidemargin}{-0.25cm}
\addtolength{\headwidth}{0.5cm}
\addtolength{\textwidth}{0.5cm}

%---------- Entries
\def \entryMargin{1em}

%---------- Icons
\def \iconWidth{1.5em}
% Predefined icons, based on FontAwesome.
% see https://ctan.mirror.rafal.ca/fonts/fontawesome/doc/fontawesome.pdf for more options.
\def \linkedinIcon{\icon{\faLinkedin}}
\def \phoneIcon{\icon{\faPhone}}
\def \homeIcon{\icon{\faHome}}
\def \emailIcon{\icon{\faEnvelope}}
\def \githubIcon{\icon{\faGithub}}
\def \websiteIcon{\icon{\faGlobe}}

\def \educationIcon{\icon{\faGraduationCap}}
\def \workIcon{\icon{\faBriefcase}}
\def \projectsIcon{\icon{\faFlask}}
\def \communicationIcon{\icon{\faComments}}
\def \awardsIcon{\icon{\faTrophy}}
\def \skillsIcon{\icon{\faGears}}
\def \interestsIcon{\icon{\faGamepad}}

\def \expertIcon{\icon{\faStar}}
\def \proficientIcon{\icon{\faStarHalfFull}}
\def \noviceIcon{\icon{\faStarO}}
\def \personalIcon{\icon{\faUser}}

\def \bulletIcon{\icon{\faAngleRight}}
\def \accentIcon{\icon{\faAngleDoubleRight}} % \faCaretRight \faAngleDoubleRight \faCode


%-------------------------------------------------------------------------------------------------
%---------- DATA HERE
%-------------------------------------------------------------------------------------------------
%---------- Header data
\def \name{Filipi Maciel}
\def \nameSuffix{}
\def \subtitleText{Desenvolvedor de Software}
\def \summaryText{Backend | C\# | .Net Core |}

\def \linkedinLink{https://www.linkedin.com/}
\def \linkedinText{/filipi-maciel-891300132}

\def \phoneText{55+ 38 99216-0517}

\def \homeText{Montes Claros - MG}

\def \emailLink{mailto:filipijardimrm@gmail.com}
\def \emailText{filipijardimrm@gmail.com}

\def \githubLink{https://github.com/Filip3ra}
\def \githubText{/Filip3ra}

\def \websiteLink{}
\def \websiteText{}

%---------- Header format
\def \fullName{\textbf{\huge\name}\large\hspace{0.3em}\textit{\nameSuffix}}
\def \subtitle{\textit{\small\cvText{\subtitleText}}}
\def \summary{\cvText{\summaryText}}
\def \linkedin{\small\linkedinIcon\href{\linkedinLink}{\linkedinText}}
\def \phone{\small\phoneIcon{\phoneText}}
\def \home{\small\homeIcon{\homeText}}
\def \email{\small\emailIcon\href{\emailLink}{\emailText}}
\def \github{\small\githubIcon\href{\githubLink}{\githubText}}
\def \website{\small\websiteIcon\href{\websiteLink}{\websiteText}}


%-------------------------------------------------------------------------------------------------
%---------- START
%-------------------------------------------------------------------------------------------------
\begin{document}
%-------------------------------------------------------------------------------------------------
%---------- HEADER
%-------------------------------------------------------------------------------------------------
\fancyhead[L]{
    \begin{tabular}[c]{l}
        {\fullName}\\
        {\subtitle}
    \end{tabular}
    \vspace{-0.75em}
}
\fancyhead[R]{
    \begin{tabular}[c]{l@{\hspace{1em}}l@{\hspace{1em}}l}
        % Configure the order in which the header data appears. Must be in 3 colums.
        {\phone} & {\github} & {\email} \\
        {\home} & {\website} & {\linkedin}\\
        \vspace{0.5em}
    \end{tabular}
    \vspace{-0.75em}
}
%-------------------------------------------------------------------------------------------------
%---------- INTRODUCTION
%-------------------------------------------------------------------------------------------------
\summary\\ % Comment out if not using
%-------------------------------------------------------------------------------------------------
%---------- FORMAÇÃO ACADÊMICA
%-------------------------------------------------------------------------------------------------
\section{Formação Acadêmica}{\educationIcon}
\cvEntryNTPL{Instituto Federal de Educação, Ciência e Tecnologia do Norte de Minas Gerais - IFNMG}{2019 -- 2025}{Bacharelado em Ciência da Computação}{Montes Claros - Minas Gerais, Brasil}
%\cvEntryNTPL{Nome da Escola/Universidade}{Início -- Fim}{Curso}{Localização}

%-------------------------------------------------------------------------------------------------
%---------- EXPERIÊNCIA PROFISSIONAL
%-------------------------------------------------------------------------------------------------
\section{Experiência Profissional}{\workIcon}
\cvEntryNTPLD
    {Sicoob Credinosso}{Nov 2022 – Fev 2024}
    {Analista de Dados – Inteligência de Negócios no Setor Financeiro}{Montes Claros, Minas Gerais, Brasil}{
    \cvItemS{VBA | Desenvolvimento de Software | Excel | IBM Cognos Analytics | Power BI }
    \cvItem{Responsável por otimizar rotinas e automatizar relatórios utilizando VBA (Visual Basic). Extração, processamento e gerenciamento de dados com IBM Cognos Analytics.}
    \cvItem{Realizei manutenção em sistemas desenvolvidos em VBA, incluindo sistemas de gestão de arquivos e acompanhamento de metas. Desenvolvi e mantive dashboards interativos no Power BI.}
    \cvItem{Apresentação de resultados utilizando ferramentas do Microsoft Office, como Word e PowerPoint.}
}
\cvEntryNTPLD
    {UaiCode (Empresa Júnior)}{Set 2016 – Jun 2017}
    {Gerente de Projetos – Desenvolvimento de Software}{Montes Claros, Minas Gerais, Brasil}{
    \cvItemS{SCRUM | Trello | Planejamento | Gestão de Equipes}
    \cvItem{Gerenciei todo o ciclo de vida de projetos de desenvolvimento de software, do planejamento à entrega. Atuei como elo entre stakeholders, desenvolvedores e equipes de QA para garantir o alinhamento estratégico.}
    \cvItem{Supervisionei cronogramas, custos e gerenciamento de riscos utilizando Trello. Implementei a metodologia ágil Scrum e práticas de DevOps para automatizar fluxos de trabalho de desenvolvimento.}
    %\cvItem{Descrição de tarefas 3}
}
\cvEntryNTPLD
{CEAD - Centro de Educação a Distância}{Nov 2016 – Jun 2019}
{Editor e Produtor de Vídeo}{Montes Claros, Minas Gerais, Brasil}{
	\cvItemS{After Effects | Photoshop | Maya | Adobe Premiere Pro CC}
	\cvItem{Como editor, utilizei software Premiere Pro CC para edição de aulas de ensino a distância, realizando correções de áudio, tratamento de imagem. Explorei recursos e funcionalidades do software Adobe After Effects, Maya e Blender, realizando modelagens 3D para vídeos e propagandas institucionais do IFNMG.}
	\cvItem{Como produtor, capacitei professores para as gravações e operei equipamentos de filmagem como câmeras, microfones lapelas, iluminação, lousa interativa, teleprompter, mesa de som, T2 iddr playout.}
	%\cvItem{Descrição de tarefas 3}
}
%-------------------------------------------------------------------------------------------------
%---------- PROJETOS PESSOAIS
%-------------------------------------------------------------------------------------------------
%\newpage %Descomente se esta seção não estiver inteiramente nesta página
\section{Projetos Pessoais}{\projectsIcon}
%\cvEntryND{\cvItemND{Inglês, Francês, Espanhol, Português}{}}


\cvEntryNTD{Gerenciamento de Chamados}{Mai 2025 – Mai 2025}{
	\cvItemS{ PHP | Laravel | MVC | MySql | Blade }
	\cvItem{Esse sistema simula o gerenciamento de chamados dentro de um departamento. Um chamado representa uma demanda feita por um usuário e passa por diferentes estados ao longo do seu ciclo de vida (como Pendente, Em andamento, Concluído, etc). Usuários autenticados, com as devidas permissões, podem acompanhar e atualizar o andamento dessas demandas, tornando possível simular de forma prática o fluxo de atendimento e resolução de chamados.}
	%\cvItem{\href{https://investimento-api.vercel.app/}{Disponível aqui.}}	
}


\cvEntryNTD{Sistema de Gerenciamento de Eventos}{Abr 2025 – Abr 2025}{
	\cvItemS{ C\# | ASP.NET | .NET Framework | Angular | SQLite | Banco de Dados | Minimal API }
	\cvItem{Sistema de gerenciamento de eventos desenvolvido utilizando C\#, .NET e Minimal API. O projeto inclui login com autenticação e operações de CRUD para eventos e funcionários associados a esses eventos. O sistema foi projetado para facilitar a organização de eventos, com foco em festas e reuniões de pequeno e médio porte.}
	%\cvItem{\href{https://investimento-api.vercel.app/}{Disponível aqui.}}	
}

\cvEntryNTD{API de Investimento}{Abr 2025 – Abr 2025}{
	\cvItemS{ JavaScript | HTML | CSS }
	\cvItem{"De 0 aos 100K" é uma API construída com JavaScript e HTML que simula uma calculadora de investimentos. Ela estima quanto tempo levaria para atingir cem mil reais por meio de investimentos mensais e anuais. Este projeto permitiu criar uma aplicação simples, porém funcional, do zero — abrangendo a ideia inicial, controle de versão, desenvolvimento e implantação. Serve como Prova de Conceito (PoC) para avaliar a viabilidade de uma plataforma de simulação de investimentos em larga escala.}
	\cvItem{\href{https://investimento-api.vercel.app/}{Disponível aqui.}}	
}



%-------------------------------------------------------------------------------------------------
%---------- PROJETOS UNIVERSITÁRIOS
%-------------------------------------------------------------------------------------------------
%\newpage %Descomente se o PRIMEIRO ITEM desta seção não estiver inteiramente nesta página
\section{Projetos Universitários}{\projectsIcon}
\cvEntryNTD{My Code: Programação para Crianças}{Jan 2022 – Dez 2022}{
	\cvItemS{Programação | Robótica | Ensino | Monitoria }
	\cvItem{Atuei como instrutor ensinando programação e robótica para crianças utilizando um kit de robótica personalizado desenvolvido pela equipe do projeto. O kit incluía componentes eletrônicos programáveis, como Arduino, sensores e LEDs. Os alunos programavam o kit por meio da plataforma DB4K. Foram criadas atividades e exercícios interativos para proporcionar uma experiência de aprendizado divertida e prática.}
}
\cvEntryNTD{My Maker: Programação e Robótica}{Jan 2021 – Dez 2022}{
	\cvItemS{Programação | Robótica | Matemática | Ensino | Monitoria }
	\cvItem{Atuei como instrutor em aulas de programação e robótica voltadas para crianças e adolescentes. As aulas incluíram programação em blocos utilizando o Code.org, projetos de robótica com uso do Tinkercad para design de circuitos eletrônicos, além de atividades "desplugadas" para ensinar conceitos básicos de programação e matemática.}
}
\cvEntryNTD{Programação e Robótica para Crianças e Adolescentes}{Jul 2020 – Dez 2020}{
	\cvItemS{Programação | Eletrônica | Robótica | Ensino | Monitoria }
	\cvItem{Atuei como instrutor em aulas de programação e robótica utilizando materiais recicláveis. Criei e gravei videoaulas educativas para crianças e adolescentes com o objetivo de introduzi-los à programação e à robótica de forma criativa e acessível.}
}
%\newpage
\cvEntryNTD{Program Children: Programação para Crianças}{Jun 2017 – Dez 2017}{
	\cvItemS{ Programação | Matemática | Lógica }
	\cvItem{Ministrei aulas de programação com foco no desenvolvimento do pensamento lógico e computacional em crianças. O objetivo era introduzir conceitos da ciência da computação, despertar o interesse pela tecnologia e melhorar o desempenho em matemática.}
}

\cvEntryNTD{Dicionário Animado de LIBRAS Aplicado ao Ensino de Matemática: Tecnologia para Inclusão}{Mai 2016 – Dez 2016}{
	\cvItemS{ LIBRAS | Modelagem 3D | Animação | Matemática }
	\cvItem{Atuei com modelagem e animação 3D em Língua Brasileira de Sinais (LIBRAS) utilizando o software Blender. Desenvolvi conteúdos animados para um dicionário focado em símbolos matemáticos, promovendo acessibilidade e inclusão na educação.}
}

\cvEntryNTD{Iniciativa GeoGebra: Uso do GeoGebra no Ensino de Cálculo}{Mar 2015 – Out 2015}{
	\cvItemS{ Applets | GeoGebra | Cálculo | Matemática }
	\cvItem{Desenvolvi applets interativos utilizando o GeoGebra para apoiar o ensino de cálculo diferencial e integral para alunos de graduação.}
}

%-------------------------------------------------------------------------------------------------
%---------- VOLUNTARIADO
%-------------------------------------------------------------------------------------------------
%\newpage %Descomente se esta seção não estiver inteiramente nesta página
\section{Trabalho Voluntário}{\skillsIcon}
\cvEntryND{
	\cvItemND{Diretor de Projetos}{Enactus IFNMG – Fui membro da equipe Enactus-IFNMG, onde atuei em projetos sociais voltados para o ensino de programação para crianças. Tive a oportunidade de aprender sobre gestão de projetos, ferramentas de gerenciamento, relações interpessoais, importância do trabalho social e do trabalho em equipe. Mai 2016 – Dez 2016}
}

%-------------------------------------------------------------------------------------------------
%---------- COMUNICAÇÃO
%-------------------------------------------------------------------------------------------------
%\newpage %Descomente se esta seção não estiver inteiramente nesta página
\section{Comunicação}{\communicationIcon}
%\cvEntryND{\cvItemND{Inglês, Francês, Espanhol, Português}{}}
\cvEntryND{
	\cvItemL{Inglês}{Profissional}{Português}{Nativo}{}{}{}{}
}

%-------------------------------------------------------------------------------------------------
%---------- PUBLICAÇÕES
%-------------------------------------------------------------------------------------------------
%\newpage %Descomente se o PRIMEIRO ITEM desta seção não estiver inteiramente nesta página
\section{Publicações}{\awardsIcon}
\cvEntryNTD{ccessnet: Dicionário Animado em LIBRAS}{26 Jun, 2017}{
	\cvItemS{Revista Intercâmbio Unimontes}
}
\cvEntryNTD{Criptografia Caótica: Transformação do Gato de Arnold Aplicada à Segurança da Informação}{1 Mai, 2016}{
	\cvItemS{SIC - Seminários de Iniciação Científica}
}

\begin{comment}
%-------------------------------------------------------------------------------------------------
%---------- SKILLS
%-------------------------------------------------------------------------------------------------
%\newpage %Uncomment if this section is not entirely on this page
\section{Skills}{\skillsIcon}
\cvEntryND{
    \cvItemND{Category}{Comma separated list of skills}
    \cvItemND{Category}{Comma separated list of skills}
    \cvItemND{Category}{Comma separated list of skills}
    \cvItemND{Category}{Comma separated list of skills}
}

%-------------------------------------------------------------------------------------------------
%---------- INTERESTS
%-------------------------------------------------------------------------------------------------
%\newpage %Uncomment if this section is not entirely on this page
\section{Interests}{\interestsIcon}
\cvEntryND{
    \cvItemND{Category}{Description of what interests you in this category}
    \cvItemND{Category}{Description of what interests you in this category}
    \cvItemND{Category}{Description of what interests you in this category}
    \cvItemND{Category}{Description of what interests you in this category}
}


\end{comment}


\end{document}