\documentclass[proposal]{ppgccufmg} % utilizem 'tcc' para a monografia final ou 'proposal' para projeto de TCC

% configurações do documento
\usepackage[brazil]{babel}      % se o documento for em português
%\usepackage[latin1]{inputenc}
\usepackage[utf8]{inputenc}		% devo usar esse pacote ?  
\usepackage[T1]{fontenc}
\usepackage{graphicx}
\usepackage[
portuguese,
bookmarks=true,
bookmarksnumbered=true,
linktocpage,
colorlinks,
citecolor=black,
urlcolor=blue,
linkcolor=blue,
filecolor=black,
]{hyperref}
\usepackage[square]{natbib}

% para pseudo códigos
\usepackage{amsmath}
\usepackage[linesnumbered,ruled]{algorithm2e}

% para ajustar imagens grandes
\usepackage[export]{adjustbox}

% para usar o diagrama de gantt
\usepackage{pgfgantt}
\usepackage{xcolor}
\usepackage{tikz}

% para comentários de multiplas linhas
\usepackage{comment}

% Para arrays
\usepackage{amsmath}
\usepackage{array}

% Usado nos topicos finais
\usepackage{algorithmic}

% Pacote necessário para colorir as células
\usepackage{colortbl}

% referencia de chapters e sections
\usepackage{nameref}

% tabelas de resultados
\usepackage{geometry}
\usepackage{longtable}
\geometry{a4paper, margin=1in}
\usepackage{placeins}

% quebrar linha dentro do cebeçalho
\usepackage{makecell}

% forçar texto depois da tabela
\usepackage{afterpage}


\begin{document}
	
	% O comando a seguir, \ppgccufmg, provê todas as informações relevantes para a
	% classe ppgccufmg. Por favor, consulte a documentação para a descrição de
	% cada chave.
	
	% Um exemplo para documentos em português é apresentado a seguir:
	\ppgccufmg{
		title={Método Gráfico e Algoritmo Genético com Chaves Aleatórias Enviesadas Aplicado ao Job-Shop Scheduling Just-in-Time},
		authorrev={Rodrigues Jardim, Filipi Maciel},
		cutter={D1234p},   % dados que futuramente serão utilizados pela biblioteca
		cdu={519.6*82.10}, % dados que futuramente serão utilizados pela biblioteca (!)
		university={Instituto Federal do Norte de Minas Gerais},
		campus={Montes Claros},
		coursetype = {Bacharelado},
		course={Ciência da Computação},
		address={Montes Claros},
		date={2024-09},
		keywords={Job-Shop, Scheduling, Just-in-Time, JSSP},
		advisor={Tadeu Knewitz Zubaran},
		%approval={img/approvalsheet.eps},  (!)
		abstract={Resumo}{resumo},
		abstract=[english]{Abstract}{abstract},
		dedication={dedicatoria}, 
		%ack={agradecimentos},
		epigraphtext={Um bom programador é alguém que sempre olha para os dois lados antes de atravessar uma rua de mão única.}{Doug Linder},
     	}
	
	% LISTA DE EPIGRAFES 
	% "Um bom programador é alguém que sempre olha para os dois lados antes de atravessar uma rua de mão única." - Doug Linder, cientista da computação
	
	%Se os carpinteiros construíssem edifícios da mesma forma que os programadores fazem programas, o primeiro pica-pau que aparecesse destruiria toda a civilização. - Autor Desconhecido
	
	% It's not darkness that deceives and hides. It's light - Death Note
	% O que é melhor, nascer bom ou superar sua natureza maligna com muito esforço? - Paarthurnax, The Elder Scrolls 5: Skyrim
	% The right man in the wrong place can make all the difference in the world. - G-Man, Half-Life 2
	% Stand in the ashes of a trillion dead souls, and asks the ghosts if honor matters. 
	% The silence is your answer. - Javik, Mass Effect 3
	% The harder you beat a man, the taller he stands. - The Jackal, Far Cry 2
	% 
	
	% [X] Definir configurações
	\input{configuracoes}
	
	% TÍTULO
	% CONTRA CAPA
	% DEDICATÓRIA
	% AGRADECIMENTO
	% EPÍGRAFO
	
	% [X] Escrever rersumo
	%\input{resumo}
	%\input{abstract}
	
	
	% SUMÁRIO
	% LISTA DE FIGURAS
	% LISTA DE TABELAS 
	% TODO- Após o lista de tabelas tem um monte de texto de outros arquivos, como resolver? 
	
	% [X] Escrever intro
	\input{Introducao}
	
	% [~] Escrever desenvolvimento
	\input{Desenvolvimento}
	
	% [~] Escrever sobre trabalhos relevantes
	\input{Trabalhos Correlacionados}
		
	% [ ] Escrever a metodologia
	\input{Metodologia}
	
	% Esses dois só são escritos após a defesa
	\input{Resultados}
	\input{Conclusao}
	
	% Aqui vem a parte da bibliografia: use o comando \ppgccbibliography indicando
	% apenas o nome do arquivo .bib (sem a extensão).
	\ppgccbibliography{bibfile}
	

	
	% Este comando encapsula o conjunto de apêndices. A sua função é fazer com que
	% a numeração dos apêndices seja feita com letras maiúsculas (A, B, C, etc.) e
	% a palavra "Apêndice" anteceda as entradas no Sumário.
	\begin{appendices}
		\input{ApendiceA}
		%\input{ApendiceB}
	\end{appendices} % Fim dos apêndices (usar apenas depois do último apêndice)
	
	
	% Este comando encapsula o conjunto de anexos. A sua função é fazer com que a
	% numeração dos anexos seja feita com letras maiúsculas (A, B, C, etc.) e a
	% palavra "Anexo" anteceda as entradas no Sumário.
%	\begin{attachments}
%		\input{AnexoA}
%		\input{AnexoB}
%	\end{attachments} % Fim dos anexos (usar apenas depois do último anexo)
	
	
\end{document}
